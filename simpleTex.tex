%新文件C-c C-e 开始一个环境,会有提示建立什么样的文档
%C-c C-c 编译tex文件,C-c C-v预览pdf文件
\documentclass{article}% class 有article, book, report, letter...可以选择
\usepackage{times}
\usepackage{graphicx}
%\usepackage{minted}
\graphicspath{{./figs/}{./}}
\begin{document}%文章开始
%C-c C-m 提示输入Macro也就是LaTex命令
\title{How to Structure a \LaTeX{} Document} %\LaTeX is a macro for printing the Latex logo
\author{wenxing shen\\%\\代表强制换行
  \emph{dlut.xing@gmail.com} %emph为emphasize,表示强调
}
\date{\today}
%title, author, date 一般叫做文章的top matter
\maketitle %maketitle 为排版top matter,不要标题可以省略这一句
%C-c C-e 开始一个环境,即有begin和end包起来的段落
\begin{abstract}
  一个简单的tex制作文档
\end{abstract}
\section{Introduction}%C-c C-s 开始一个section
\label{sec:intro}
This small document is designed to illustrate how easy it is to create a
well structured document within \LaTeX\cite{lamport94}.  You should quickly be able to
see how the article looks very professional, despite the content being
far from academic.  Titles, section headings, justified text, text
formatting etc., is all there, and you would be surprised when you see
just how little markup was required to get this output.

\section{Structure}
One of the great advantages of \LaTeX{} is that all it needs to know is
the structure of a document, and then it will take care of the layout
and presentation itself.  So, here we shall begin looking at how exactly
you tell \LaTeX{} what it needs to know about your document.
\begin{figure}
  \centering
   \includegraphics[width=1in]{miband_offline.png}
  \caption{logo}
  \label{fig:logo}
\end{figure}

Figure~ \ref{fig:logo} is the famous mi band Logo!
\subsection{Top Matter}
The first thing you normally have is a title of the document, as well as
information about the author and date of publication.  In \LaTeX{} terms,
this is all generally referred to as \emph{top matter}.

\subsubsection{Article Information}
%Set up an 'itemize' environment to start a bulleted list.  Each
%individual item begins with the \item command.  Also note in this list
%that it has two levels, with a list embedded in one of the list items.
%文章的章节次序是1.chapter 2.section 3.subsection 4.subsection 5.paragraph 6. subparagraph
%其中chapter是给book和report用的,article从section开始
%itemize 不带序号的列表
\begin{itemize}
\item \verb|\title| -- the title of article% \verb||为verbatim,原样输出||中间的内容
\item \verb|\date| -- the date %M-ret回车并新起一个item
  \begin{itemize}
  \item \texttt{\textbackslash date\{\textbackslash today\}} - to get the
    date that the document is typeset.
  \item \texttt{\textbackslash date\{\emph{date}\}} - for a specific date.
  \item \texttt{\textbackslash date\{\}} - for no date.%\verb|\date{}|
  \item \verb|\date{}| - no date
  \end{itemize}
\end{itemize}

\subsection{Sectioning Commands}
\label{sec:sectioning-commands}
The commands for inserting sections are fairly intuitive.  Of course,
certain commands are appropriate to different document classes.  For
example, a book has chapters but a article doesn't.
\begin{center}
  \begin{tabular}{|l|c|r|}%l代表左对齐,c代表居中,r代表右对齐
    \hline % 画一条横线
    Command & Level & Info \\ %&为两列之间的分隔符
    \hline
    \verb|\part{}| & -1 & part\\
     \verb|\chapter{}| & 0 &chapter\\
     \verb|\section{}| & 1 &section\\
     \verb|\subsection{}| & 2 &subsection\\
     \verb|\subsubsection{}| & 3 &subsubsection\\
     \verb|\paragraph{}| & 4 & paragraph\\
     \verb|\subparagraph{}| & 5 &subparagraph\\
     \hline
  \end{tabular}
\end{center}
Numbering of the sections is performed automatically by \LaTeX{}, so
don't bother adding them explicitly, just insert the heading you want
between the curly braces.  If you don't want sections number, then add
an asterisk (*) after the section command, but before the first curly
brace, e.g., \verb|section*{A Title Without Numbers}|.

\section{Item}
\label{sec:item}

% 美元符号($)在LaTeX里面是特殊字符。夹在两个美元符号之间的东西,会被当
% 做数学公式来排版。每边用两个美元符号,数学公式会单独占一行。
This is a math example: $a^2=c^2+b^2$

This is a simple math example: $$c^2=a^2+b^2$$

%如果想给数学公式排序号的话,就把它放进 equation 环境里

\begin{equation}
  \label{eq:1}
  c^2 = a^2 + b^2
\end{equation}

\subsection{Insert code}
\label{sec:insert-code}
%首先要 \usepackage{minted} 。在LaTeX文件中插入代码的工具包有不少,比较传统的是 listings, 比较新潮的是 minted 。
%\begin{minted}{c}
%  #include <stdio.h>
%  int main(int argc, char *argv[])
%  {
%    printf ("Hello, world!\n");
%    return 0;
%  }
%\end{minted}
% 或者\inputminted{c}{hello.c}

\begin{thebibliography}{99}
\bibitem{lamport94} Leslie Lamport, \emph{\LaTeX: A Document
    Preparation System}.  Addison Wesley, Massachusetts, 2nd Edition,
  1994.
\bibitem{wikibooks}
  http://en.wikibooks.org/wiki/LaTeX/simple.tex
\end{thebibliography}
\end{document}
