\documentclass[UTF8]{ctexart}
%ctexart ctexbook ctexrep
% CTeX包提供了较为完善的中文支持,排版的方案(例如段落缩进)也符合中国
% 人的习惯,但是CTeX是基于Windows设计的,在移植到Mac时会出现一些小问题。
% 主要是字体的问题,CTeX中默认使用的SimSun等字体在Mac OS中并不存在,取
% 而代之的是“华文宋体”等华文系列的字体。因此如果不配置,会因找不到字体
% 而出现编译错误。另外,在Windows中常用的“隶书”和“幼圆”两种字体,在Mac
% OS中根本不存在,也没有可以替换的字体。

% 安装好字体后需要修改一下配置文件,在应用程序中打开终端,输入下列命令
% sudo vim /usr/local/texlive/2011/texmf-dist/tex/latex/ctex/fontset/ctex-xecjk-winfonts.def
%输入密码后将其内容替换为:

%\setCJKmainfont[BoldFont={STHeiti},ItalicFont=STKaiti]
%  {STSong}
%\setCJKsansfont{STHeiti}
%\setCJKmonofont{STFangsong}

%\setCJKfamilyfont{zhsong}{STSong}
%\setCJKfamilyfont{zhhei}{STHeiti}
%\setCJKfamilyfont{zhkai}{STKaiti}
%\setCJKfamilyfont{zhfs}{STFangsong}
%\setCJKfamilyfont{zhli}{LiSu}
%\setCJKfamilyfont{zhyou}{YouYuan}

%\newcommand*{\songti}{\CJKfamily{zhsong}} % 宋体
%\newcommand*{\heiti}{\CJKfamily{zhhei}}   % 黑体
%\newcommand*{\kaishu}{\CJKfamily{zhkai}}  % 楷书
%\newcommand*{\fangsong}{\CJKfamily{zhfs}} % 仿宋
%\newcommand*{\lishu}{\CJKfamily{zhli}}    % 隶书
%\newcommand*{\youyuan}{\CJKfamily{zhyou}} % 幼圆

%\endinput
%下面我们来测试配置,使用TeXShop作为IDE,代码存为UTF-8编码,以XeLaTeX编译。
% 参考文献 http://kqwd.blog.163.com/blog/static/4122344820117725633776/

% 使用普通宏包设置
%\documentclass{article}
%\usepackage{ctex}
%效果和前面相同
\begin{document}
\title{使用中文}
\maketitle
第一篇中文pdf
\begin{center}
1. 字体示例:\\
  \begin{tabular}{c|c}
    \hline
    \textbf{\TeX 命令} & \textbf{效果}\\
    \hline
    \verb|{\songti 宋体}| & {\songti 宋体}\\
    \hline
    \verb|{\heiti 黑体}| & {\heiti 黑体}\\
    \hline
    \verb|{\fangsong 仿宋}| & {\fangsong 仿宋}\\
    \hline
    \verb|{\kaishu 楷书}| & {\kaishu 楷书}\\
    \hline
  \end{tabular}
\end{center}
\begin{center}
  2. 字号示例:\\
  {\zihao{0}初号}
  {\zihao{1}一号}
  {\zihao{2}二号}
  {\zihao{3}三号}
  {\zihao{4}四号}
  {\zihao{5}五号}
  {\zihao{6}六号}
  {\zihao{7}七号}
  {\zihao{8}八号}
\end{center}
\end{document}
