% Created 2015-10-05 Mon 11:41
\documentclass[11pt]{article}
\usepackage[utf8]{inputenc}
\usepackage[T1]{fontenc}
\usepackage{fixltx2e}
\usepackage{graphicx}
\usepackage{longtable}
\usepackage{float}
\usepackage{wrapfig}
\usepackage{soul}
\usepackage{textcomp}
\usepackage{marvosym}
\usepackage{wasysym}
\usepackage{latexsym}
\usepackage{amssymb}
\usepackage{hyperref}
\tolerance=1000
\providecommand{\alert}[1]{\textbf{#1}}

\title{emacs learn}
\author{shenwenxing@xiaomi.com}
\date{\today}
\hypersetup{
  pdfkeywords={},
  pdfsubject={},
  pdfcreator={Emacs Org-mode version 7.9.3f}}

\begin{document}

\maketitle

\setcounter{tocdepth}{3}
\tableofcontents
\vspace*{1cm}
\label{content-position}
\section{To-Do 模式}
\label{sec-1}
\subsection{emacs 中的To-Do}
\label{sec-1-1}


To-do可以用来做日程安排,todo,done,top分别表示要完成的事情,已完成的事情,和优先事情。

首先要修改.emacs文件

  (setq todo-file-do ``\~{}/.emacs.d/todo-do'')

  (setq todo-file-done ``\~{}/.emacs.d/todo-done'')

  (setq todo-file-top ``\~{}/.emacs.d/todo-top'')

在\~{}/.emacs.d/文件夹下建立todo-do文件

在todo-do文件中建立自己的category
M-x todo-add-category
输入分类的名称,如work,study,travel等

切换到todo模式
M-x todo-mode进入todo mode。

下面是一些快捷操作
i    加入一个条目
e    编辑条目
k    删除条目
+/-  在不同category之间切换
j    跳转到不同的category
f    对已完成的事情进行归档,并可以进行评论。
\subsection{org模式中的To-Do}
\label{sec-1-2}
\subsubsection{\textbf{TODO} 学习todo模式}
\label{sec-1-2-1}

  \texttt{SCHEDULED:} \textit{2014-10-26 Sun}

在org模式下,只需要CS-RET即可添加TODO,屏幕上会显示''* TODO'',直接在后面添加内容即可。
C-c C-s可以添加时间
C-c C-t可以改变TODO的状态
S-right/left也可以改变TODO的状态
\subsection{\textbf{TODO} 学习思维导图}
\label{sec-1-3}

   \texttt{SCHEDULED:} \textit{2014-10-26 Sun}
\subsubsection{todo中的命令}
\label{sec-1-3-1}

CS-RET 添加同级的TODO命令
\subsubsection{\textbf{TODO} 聚会}
\label{sec-1-3-2}

    \texttt{SCHEDULED:} \textit{2014-10-26 Sun}
    
\section{文件操作}
\label{sec-2}

  C-x C-w 将文件另存为
  C-x i   将文件插入光标当前位置
  C-x C-r 以只读的方式打开文件
  
\section{缓冲区}
\label{sec-3}


  C-x s 保存所有的缓冲区C-z 将emacs挂起,然后回到shell中,并不退出
  emacs,然后可以用\%emacs或者fg回到emacs
  
\section{使用帮助}
\label{sec-4}

  C-h c 快捷键: 显示快捷键的简要说明
  C-h k 快捷键: 显示快捷家所对应的命令的详细说明
  C-h a 关键字: 显示包含指定关键字的命令
  C-h i      :  查看info文档
  M-x ielm      打开ielm,输入exec-path可以查看emacs的执行m路径
  C-h b 显示当前缓冲区所有可用的快捷键
  C-h m 查看当前模式
  C-h f 显示函数的功能
  C-M v 向下滚动另一个窗口
  C-- C-M v 向上移动另一个窗口
\section{在emacs中运行shell命令}
\label{sec-5}

  M-! cmd RET 将结果输出到新的窗口
  C-u M-! cmd RET 将结果输出到光标所在的位置
  
\section{org-mode}
\label{sec-6}
\subsection{表格}
\label{sec-6-1}

\begin{table}[htb]
\caption{name-value} \label{table1}
\begin{center}
\begin{tabular}{lrr}
 name  &  value  &   sd  \\
 jack  &      1  &  2.1  \\
 mike  &      2  &  3.2  \\
\end{tabular}
\end{center}
\end{table}



H$_{2}$O
\texttt{git}
\subsection{列表}
\label{sec-6-2}

\begin{itemize}
\item 无序列表通常以'-','+','*'开头
\item 有序列表通常以'1.','1)'开头
\item 描述列表用'::'
\end{itemize}
:: 仅仅是一个列表描述
\subsubsection{Lord of the Rings}
\label{sec-6-2-1}

   My favorite scenes are (in this order)
\begin{enumerate}
\item The attack of the Rohirrim
\item Eowyn's fight with the witch king
\begin{itemize}
\item this was already my favorite scene in the book
\item I really like Miranda Otto.
\end{itemize}
\item Peter Jackson being shot by Legolas
\begin{itemize}
\item on DVD only
\end{itemize}
He makes a really funny face when it happens.
\end{enumerate}

ORG-LIST-END-MARKER
   But in the end, no individual scenes matter but the film as a whole.
   Important actors in this film are:
\begin{description}
\item[Elijah Wood] He plays Frodo
\item[Sean Austin] He plays Sam, Frodo's friend.  I still remember
     him very well from his role as Mikey Walsh in The Goonies
\end{description}
ORG-LIST-END-MARKER
\subsection{大纲之间移动}
\label{sec-6-3}

C-c C-n/p  标题移动
C-c C-f/b  标题 同级别移动
C-c C-u    跳到上一级标题
C-c C-j    切换到大纲预览状态


\begin{center}
\begin{tabular}{ll}
 快捷键          &  说明                                        \\
 M-RET           &  插入同级别标题                              \\
 M-S-RET         &  插入同级别TODO标题                          \\
 M-LEFT/RIGHT    &  将当前标题升级/降级                         \\
 M-S-LEFT/RIGHT  &  将子树升级/降级                             \\
 M-UP/DOWN       &  将子树上/下移动                             \\
 C-c C-w         &  将子树或者区域移动到另一个标题处(跨缓冲区)  \\
 C-c c-x b       &  在新缓冲区显示当前分支                      \\
 C-c /           &  只列出包含搜索结果的大纲                    \\
\end{tabular}
\end{center}
\subsection{嵌入数据}
\label{sec-6-4}



\begin{verbatim}
s    #+begin_src ... #+end_src 
e    #+begin_example ... #+end_example  : 单行的例子以冒号开头
q    #+begin_quote ... #+end_quote      通常用于引用,与默认格式相比左右都会留出缩进
v    #+begin_verse ... #+end_verse      默认内容不换行,需要留出空行才能换行
c    #+begin_center ... #+end_center 
l    #+begin_latex ... #+end_latex 
L    #+latex: 
h    #+begin_html ... #+end_html 
H    #+html: 
a    #+begin_ascii ... #+end_ascii 
A    #+ascii: 
i    #+index: line 
I    #+include: line
\end{verbatim}


\begin{verbatim}
##+BEGIN_SRC c -n -t -h 7 -w 40


##+END_SRC
c为所添加的语言
-n 显示行号
-t 清除格式
-h 设置高度
-w 设置宽度
\end{verbatim}

引用表格
\ref{table1}
\subsubsection{嵌入html}
\label{sec-6-4-1}
\subsubsection{包含文件}
\label{sec-6-4-2}

当导出文档时,你可以包含其他文件中的内容。比如,想包含你的“.emacs”文件,
你可以用:

\begin{verbatim}
##+INCLUDE: "~/.emacs" src emacs-lisp
\end{verbatim}
可选的第二个第三个参数是组织方式(例如,“quote”,“example”,或者“src”),
如果是 “src”,语言用来格式化内容。组织方式是可选的,如果不给出,文本会
被当作 Org 模式的正常处理。用 C-c ,可以访问包含的文件。

本文参考\footnote{DEFINITION NOT FOUND: fn:1 }

\hyperref[content-position]{目录}
\subsection{注脚}
\label{sec-6-5}

C-c C-x f插入注脚
C-c C-c 在脚注和内容之间切换

\end{document}
